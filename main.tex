%%%%%%%%%%%%%%%%%%%%%%%%%%%%%%%%%%%%%%%%%
% "ModernCV" CV and Cover Letter
% LaTeX Template
% Version 1.1 (9/12/12)
%
% This template has been downloaded from:
% http://www.LaTeXTemplates.com
%
% Original author:
% Xavier Danaux (xdanaux@gmail.com)
%
% License:
% CC BY-NC-SA 3.0 (http://creativecommons.org/licenses/by-nc-sa/3.0/)
%
% Important note:
% This template requires the moderncv.cls and .sty files to be in the same 
% directory as this .tex file. These files provide the resume style and themes 
% used for structuring the document.
%
%%%%%%%%%%%%%%%%%%%%%%%%%%%%%%%%%%%%%%%%%

%----------------------------------------------------------------------------------------
%	PACKAGES AND OTHER DOCUMENT CONFIGURATIONS
%----------------------------------------------------------------------------------------

\documentclass[11pt,a4paper,sans]{moderncv} % Font sizes: 10, 11, or 12; paper sizes: a4paper, letterpaper, a5paper, legalpaper, executivepaper or landscape; font families: sans or roman

\moderncvstyle{classic} % CV theme - options include: 'casual' (default), 'classic', 'oldstyle' and 'banking'
\moderncvcolor{blue} % CV color - options include: 'blue' (default), 'orange', 'green', 'red', 'purple', 'grey' and 'black'

\usepackage{lipsum} % Used for inserting dummy 'Lorem ipsum' text into the template

\usepackage[scale=0.85]{geometry} % Reduce document margins
%\setlength{\hintscolumnwidth}{3cm} % Uncomment to change the width of the dates column
%\setlength{\makecvtitlenamewidth}{10cm} % For the 'classic' style, uncomment to adjust the width of the space allocated to your name

%----------------------------------------------------------------------------------------
%	NAME AND CONTACT INFORMATION SECTION
%----------------------------------------------------------------------------------------

\firstname{Fahri Nurul Hidayat} % Your first name
% \familyname{LACHGAR} % Your last name

% All information in this block is optional, comment out any lines you don't need
%\title{Curriculum Vitae}
\address{}{Jakarta, Indonesia 13790}
\mobile{(+62) 853 1997 1945}
%\phone{(000) 111 1112}
%\fax{(000) 111 1113}
\email{fahri.cyber@gmail.com}
\homepage{https://fahrinh.github.io}{https://fahrinh.github.io} % The first argument is %the url for the clickable link, the second argument is the url displayed in the %template - this allows special characters to be displayed such as the tilde in this %example
%\extrainfo{additional information}
%\photo[70pt][0.4pt]{picture} % The first bracket is the picture height, the second is %the thickness of the frame around the picture (0pt for no frame)
\quote{A passionate software developer}

%----------------------------------------------------------------------------------------

\begin{document}

\makecvtitle % Print the CV title

%----------------------------------------------------------------------------------------
%	EDUCATION SECTION
%----------------------------------------------------------------------------------------

\section{Education}

\cventry{2014--2017}{Universitas Indonesia (UI) - Master's degree in Computer Science}{}{}{}{GPA: 3.65\\
Master's Thesis: "Generating Software Product Line Family on Go Programming Language Based on Abstract Behavior Specification (ABS) Modelling Language"}  % Arguments not required can be left empty

\cventry{2007--2011}{Universitas Indonesia (UI) - Bachelor's degree in Computer Science}{}{}{}{GPA: 3.32\\
Undergraduate Thesis: "Development of Client-Server Matlab Application Based on RabbitMQ Messaging"}  % Arguments not required can be left empty


%----------------------------------------------------------------------------------------
%	WORK EXPERIENCE SECTION
%----------------------------------------------------------------------------------------

\section{Experience}

\cventry{Nov 2018--March 2019}{Senior Backend and Blockchain Developer}{Biido.Id - Marketplace and Trading Platform for Crypto Assets}{}{}
{
\begin{itemize}
  \item I developed systems to talk with crypto network nodes protocol (Ethereum \& ERC20) and payment backend system.
  \\ \textsc{Tech Stack: \textbf{Java}, \textbf{Spring Boot}, \textbf{Go}, \textbf{NodeJS}, \textbf{PostgreSQL}, \textbf{Redis}, \textbf{Docker}, \textbf{Kubernetes}}
  \item I led a mobile development team.
  \\ \textsc{Tech Stack: \textbf{Dart}, \textbf{Flutter}}
\end{itemize}
}

\cventry{May 2012--Oct 2018}{Software Developer}{Center for Computer Science of Universitas Indonesia or Pusat Ilmu Komputer Universitas Indonesia (Pusilkom UI)}{}{}
{
I worked on various projects for client companies.
\begin{itemize}
  \item Company X - \textit{Big provider of electronic transaction services} 
    \begin{itemize}
      \item May 2012--August 2013. I \textit{developed} a web based report system for client's customers with features: custom scheduling for data processing and report generation, custom report design \& custom data pipeline.\\ \textsc{Tech Stack: \textbf{Java}, \textbf{Spring MVC}, \textbf{Quartz}, \textbf{Pentaho}, \textbf{MyBatis}, \textbf{Oracle DB}}
      \item March 2014--August 2015. I led team to develop a system to integrate data from several internal systems. Users can made custom integration data workflows.\\ \textsc{Tech Stack: \textbf{Java}, \textbf{Spring MVC}, \textbf{Quartz}, \textbf{MyBatis}, \textbf{Oracle DB}}
      \item Oct 2014. I developed web services for mobile banking. On the security layer, I coded HMAC based REST auth (popularized by Amazon at that time).\\ \textsc{Tech Stack: \textbf{Java}, \textbf{Spring MVC}, \textbf{MyBatis}, \textbf{Oracle DB}}
      \item Nov 2015--June 2017. I led and developed corporate payment system for one of Indonesia's largest telecommunication services provider.\\ \textsc{Tech Stack: \textbf{Java}, \textbf{Spring Boot}, \textbf{JAX-WS} \textbf{MyBatis}, \textbf{Oracle DB}}
      \item Oct--Dec 2016. I led and developed a mobile banking application for a regional-owned bank.\\ \textsc{Tech Stack: \textbf{React Native}}
      \item Oct 2017--May 2018. I led and developed a mobile banking application with dynamic user interface.\\ \textsc{Tech Stack: \textbf{React Native}}
      \end{itemize}
  \item Product Development
  \begin{itemize}
      \item Sept--Nov 2015. I developed web services for Enterprise University Information System (EUIS).\\ \textsc{Tech Stack: \textbf{Java}, \textbf{Spring Boot}, \textbf{MyBatis}, \textbf{PostgreSQL}}
  \end{itemize}
  \item Hospital Y - \textit{Hospital of the well-known university}
  \begin{itemize}
      \item Feb--August 2017. I led and developed a hospital management system. This system is designed with highly dynamic authorization in single page application (SPA) architecture. I also built a library for generating odontogram (images of teeth)\\ \textsc{Tech Stack: \textbf{Java}, \textbf{Spring Boot}, \textbf{Vue}, \textbf{MyBatis}, \textbf{PostgreSQL}}
  \end{itemize}
\end{itemize}
}

\cventry{Feb--March 2011}{Internship}{Indonesia's National Government Internal Auditor or Badan Pengawasan Keuangan dan Pembangunan (BPKP)}{}{}
{
I \textit{developed} and \textit{made} a prototype of employee performance information system (web based application with desktop-like experience).\\
\textsc{Tech Stack: \textbf{Java}, \textbf{Vaadin}, \textbf{IBM DB2}}
}

\section{Open Source Software Contributions}
\cvitem{as Author}
{
My contributions as an author:
\begin{itemize}
    \item Odontogram: A Java library to generate odontogram
      \begin{itemize}
        \item https://github.com/pusilkom/odontogram
      \end{itemize}
    \item Azan : an Islamic prayer time application
      \begin{itemize}
        \item Azan (Cinnamon Applet) : https://cinnamon-spices.linuxmint.com/applets/view/303
        \item Azan (Gnome Shell Extension) : https://extensions.gnome.org/extension/1344/azan/
      \end{itemize}
    \item This CV
      \begin{itemize}
        \item https://github.com/fahrinh/cv
      \end{itemize}
 \end{itemize}
}
\cvitem{as Contributor}
{
My contributions as a contributor:
\begin{itemize}
    \item Abstract Behavior Specification (ABS) to Go Generator ( my master thesis' work )
    \begin{itemize}
        \item https://github.com/fahrinh/abstools
    \end{itemize}
\end{itemize}
}

\cvitem{as Reporter}
{
My contributions as a bug reporter:
\begin{itemize}
    \item I found a line-ending bug on Pentaho Reporting and I proposed a fix
    \begin{itemize}
        \item https://jira.pentaho.com/browse/PRD-4917
    \end{itemize}
\end{itemize}
}

\section{Community}
\cvitem{}
{
I am the creator and admin of the following Telegram groups:
\begin{itemize}
    \item Elixir ID -- https://t.me/elixir\_id
    \item React Native Indonesia -- https://t.me/reactnative\_id
\end{itemize}
}

\section{Certificate}
\cventry{June 2010}{Sun Certified Programmer For The Java Platform, Standard Edition 6 (SAI)}{}{}{}{}

%----------------------------------------------------------------------------------------
%	COMPUTER SKILLS SECTION
%----------------------------------------------------------------------------------------

\section{Tech Skills}

\cvitem{Basic}{Elixir, Kubernetes}
\cvitem{Intermediate}{Go, Javascript, Dart, PHP, Python, Ruby, Docker, Ansible}
\cvitem{Advanced}{Java, Linux}


%----------------------------------------------------------------------------------------
%	LANGUAGES SECTION
%----------------------------------------------------------------------------------------

\section{Languages}
\begin{small}
\cvitemwithcomment{Bahasa}{Native Speaker}{}
\cvitemwithcomment{English}{Intermediate}{}
\end{small}


%----------------------------------------------------------------------------------------
%	INTERESTS SECTION
%----------------------------------------------------------------------------------------
%\bigskip

\section{Interests}

\renewcommand{\listitemsymbol}{-~} % Changes the symbol used for lists

% \cvlistdoubleitem{Software Architecture: Domain Driven Design with Typed Functional Programming Language}{Sports: Tennis, Calisthenics)}

\cvitem{Programming Language}{Elixir}
\cvitem{Software Architecture}{Domain Driven Design (DDD) with Statically Typed Functional Programming Language (F\#, OCaml, ReasonML)}{}
\cvitem{Sports}{Tennis, Calisthenics}{}

%\cvlistdoubleitem{Robotics}{}

\section{Blog}
\cvitem{Bahasa}{https://fahrinh.github.io/}
\cvitem{English}{https://fahrinh.github.io/en/}

\section{Favourite Tech Sites}
\cvitem{}{Sites that I visit daily to get to know latest (hyped) technology or current software development issues around the world}
\cvitem{}
{
\begin{itemize}
      \item https://news.ycombinator.com/
      \item https://lobste.rs/
      \item https://www.reddit.com/r/programming
 \end{itemize}
}
%----------------------------------------------------------------------------------------
%	COVER LETTER
%----------------------------------------------------------------------------------------

% To remove the cover letter, comment out this entire block

%\clearpage

%\recipient{HR Departmnet}{Corporation\\123 Pleasant Lane\\12345 City, State} % Letter recipient
%\date{\today} % Letter date
%\opening{Dear Sir or Madam,} % Opening greeting
%\closing{Sincerely yours,} % Closing phrase
%\enclosure[Attached]{curriculum vit\ae{}} % List of enclosed documents

%\makelettertitle % Print letter title

%\lipsum[1-3] % Dummy text

%\makeletterclosing % Print letter signature

%----------------------------------------------------------------------------------------

\end{document}